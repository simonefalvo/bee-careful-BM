\section{Analisi degli investimenti}
\subsection{Tasso di sconto} 
Il tasso di sconto è stato calcolato utilizzando il modello CAPM - Capital Asset
Pricing Model, secondo cui il tasso di sconto è definito come
\begin{displaymath}
r = r_f + \beta (r_m - r_f )
\end{displaymath}
\begin{eqnarray*}
r_f &=& 0.0357 \qquad \mbox{tasso di rendimento di un titolo privo di rischi} \\ 
\beta &=& 1,24 \quad\qquad \mbox{fattore di rischio} \\
r_m &=& 0.0452 \qquad \mbox{tasso di rendimento medio del mercato}
\end{eqnarray*}

Da un’analisi del mercato italiano tecnologico abbiamo trovato come valori
$\beta=1,24$ e $r_m=4.52\%$. Come titolo risk-free è stato scelto il BTP, il cui
tasso di rendimento a 5 anni è $r_f=3.57\%$ (data di riferimento: 15/01/2019).\\
Il valore ottenuto è $r = 4.75\%$\\
Con tale tasso di sconto è possibile determinare dai flussi di cassa gli indici
VAN e TIR.

\url{https://finanza-mercati.ilsole24ore.com/azioni/indici/indici-italia-settoriali/main.php}

\subsection{Stima della domanda}
Si ipotizzata una domanda con crescita secondo una curva logistica del tipo: 
\begin{displaymath}
q(i) = \frac{1}{1 + \beta_1 e^{-i\beta_2}}
\end{displaymath}
Il numero di utenti potenziali è u=50,236 (fonte).
Avendo ipotizzato una crescita con curva logistica, poniamo $q(1)=1\%$ e
$q5=5\%$.

ARPU = ricavi medi per unità =

In un periodo di riferimento i flussi di cassa sono determinati dai ricavi
dovuti all’acquisto del prodotto da parte di nuovi clienti, pertanto gli utenti
che contribuiscono a tale calcolo vengono calcolati come $q(i) - q(i-1)$ per il
periodo $i$ di riferimento.

% tabella dcf
\begin{table}[!h]
%\centering
\begin{adjustbox}{width=\textwidth}
\begin{tabular}{c|c|r@{.}l|r@{.}l|r@{.}l|r@{.}l|c|r@{.}l}
& Nuovi 
& \multicolumn{2}{|c}{[\euro/utente]}
& \multicolumn{2}{|c}{[\euro]}
& \multicolumn{2}{|c}{[\euro]}
& \multicolumn{2}{|c|}{[\euro]}
& 
& \multicolumn{2}{|c}{[\euro]}
\\
Anno
& Utenti
& \multicolumn{2}{|c}{OPEX}
& \multicolumn{2}{|c}{Ricavi Netti}
& \multicolumn{2}{|c}{CAPEX}
& \multicolumn{2}{|c|}{CF}
& $(1+r)^{-i}$
& \multicolumn{2}{|c}{DCF}
\\

\hline
0 &      &   0&00 &      0&00 & -5080&00&  -5080&00 &1.00& -5080&00 \\
1 & 502  & 266&61 &  41861&78 &     0&00&  41861&78 &0.92& 38512&84 \\ 
2 & 253  & 386&01 &  -9110&53 &     0&00&  -9110&53 &0.84& -7652&85 \\
3 & 377  & 306&84 &  16271&32 &     0&00&  16271&32 &0.77& 12528&92 \\
4 & 558  & 254&43 &  53322&48 &     0&00&  53322&48 &0.71& 37858&96 \\
5 & 820  & 219&57 & 106952&60 &     0&00& 106952&6  &0.65& 69519&19
\end{tabular}
\end{adjustbox}
\caption{Flussi di cassa scontati}
\label{tab:dcf}
\end{table}

 





%%%%%%%%%%%%%%%%%%%%%%%%%%%%%%%%%%%%%%%%%%%%%%%%%%%%%%%%%%%%%%%%%%%%%%%%%%%%%%%%
%%%%%%%%%%%%%%%%%%%%%%%%%%%%%%%%%%%%%%%%%%%%%%%%%%%%%%%%%%%%%%%%%%%%%%%%%%%%%%%%
%%%%%%%%%%%%%%%%%%%%%%%%%%%%%%%%%%%%%%%%%%%%%%%%%%%%%%%%%%%%%%%%%%%%%%%%%%%%%%%%
\subsection{VAN e TIR}
I flussi di cassa (guadagno netto - costi ricorrenti) dall’anno t=0 a t=5 sono
rappresentati nella tabella~\ref{tab:van} insieme al VAN cumulativo.
Il VAN è calcolato come 	
\begin{displaymath}
VAN = - C_0 + \sum_{i=1}^n \frac{F_i}{(1 + r)^i}
\end{displaymath}
\begin{eqnarray*}
C_0 &=& 80000 \ \mbox{\euro} \qquad \mbox{investimento iniziale} \\
r &=& 0.0475 \qquad \mbox{tasso di sconto} \\
n &:=& \mbox{numero di periodi analizzati} \\
F_i &:=& \mbox{flusso di cassa relativo al periodo i}
\end{eqnarray*}
% tabella van
\begin{table}[!h]
\centering
%\begin{adjustbox}{width=\textwidth}
\begin{tabular}{c|r@{.}l|r@{.}l}
Anno
& \multicolumn{2}{|c}{$F_i$ [\euro]}
& \multicolumn{2}{|c}{VAN [\euro]}
\\          
\hline
0 & -5080&00 & -85080&00 \\
1 & 38512&84 & -46567&16 \\ 
2 & -7652&85 & -54220&01 \\
3 & 12528&92 & -41691&09 \\
4 & 37858&96 &  -3832&13 \\
5 & 69519&19 &  \textbf{65687}&\textbf{06} 
\end{tabular}
%\end{adjustbox}
\caption{Valore Attuale Netto}
\label{tab:van}
\end{table}

%%%%%%%%%%%%%%%%%%%%%%%%%%%%%%%%%%%%%%%%%%%%%%%%%%%%%%%%%%%%%%%%%%%%%%%%%%%%%%%%
%%%%%%%%%%%%%%%%%%%%%%%%%%%%%%%%%%%%%%%%%%%%%%%%%%%%%%%%%%%%%%%%%%%%%%%%%%%%%%%%
%%%%%%%%%%%%%%%%%%%%%%%%%%%%%%%%%%%%%%%%%%%%%%%%%%%%%%%%%%%%%%%%%%%%%%%%%%%%%%%%
\subsection{Valutazione del rischio}
I rischi individuati per il prodotto in relazione alla loro natura sono:
\begin{itemize}
\item [rischi puri:] incendi, terremoti, guasti nelle attrezzature, ...
\item [rischi speculativi:] oscillazione dei tassi di cambio, innovazioni
tecnologiche (utilizzo di droni), ...
\end{itemize}

Per osservare l’incidenza di ogni fattore di rischio si fissata una
percentuale di variazione dei parametri associati pari al 10\%. Si è quindi
calcolato il VAN sia nel caso in cui tale variazione sia positiva che negativa,
e si è determinata la sua variazione percentuale rispetto al VAN ottenuto nel
caso “reale”.

I risultati sono rappresentati nel “diagramma tornado” in figura X e nella
tabella Y, che mostrano i fattori di rischio ordinati in base alla loro
incidenza sul VAN. I fattori di rischio con incidenza maggiore sono...

\begin{table}[!h]
\centering
%\begin{adjustbox}{width=\textwidth}
\begin{tabular}{c|c|c|r@{.}l|r@{.}l}
& Valore 
&
& \multicolumn{2}{|c}{}
& \multicolumn{2}{|c}{Percentuale}
\\
Parametro
& attuale
& Variazione 
& \multicolumn{2}{|c}{VAN [\euro]}
& \multicolumn{2}{|c}{variazione}
\\

\hline
Domanda    &            &                          & 50780&78 & 271&67\%  \\
           &            &                          &-23456&60 &-271&68\%  \\
\hline                                  
Costo      & \EUR{1000} & $ + 100 \ \mbox{\euro} $ &  8441&81 & -38&21\%  \\
affitto    &            & $ - 100 \ \mbox{\euro} $ & 18885&13 & +38&22\%  \\
\hline                  
Costo      & \EUR{115}  & $+ 11.5 \ \mbox{\euro} $ & 38342&71 & 180&63\%  \\
hardware   &            & $- 11.5 \ \mbox{\euro} $ &-11016&67 &-180&63\%  \\
\end{tabular}
%\end{adjustbox}
\caption{Variazione del VAN al variare dei parametri coinvolti dai rischi}
\label{tab:risk}
\end{table}

[DIAGRAMMA TORNADO]
%%%%%%%%%%%%%%%%%%%%%%%%%%%%%%%%%%%%%%%%%%%%%%%%%%%%%%%%%%%%%%%%%%%%%%%%%%%%%%%%
%%%%%%%%%%%%%%%%%%%%%%%%%%%%%%%%%%%%%%%%%%%%%%%%%%%%%%%%%%%%%%%%%%%%%%%%%%%%%%%%
%%%%%%%%%%%%%%%%%%%%%%%%%%%%%%%%%%%%%%%%%%%%%%%%%%%%%%%%%%%%%%%%%%%%%%%%%%%%%%%%
\subsection{Modello WACC}
Per analizzare le fonti di finanziamento del nostro prodotto è stato utilizzato
il Modello WACC - Weighted Average Cost of Capital, il quale restituisce il
costo medio ponderato del capitale.
\begin{displaymath}
WACC = k_D \frac{D}{D+E} + k_E \frac{E}{D+E}
\end{displaymath}
\begin{eqnarray*}
k_D &=& 0.0475 \quad \mbox{costo del debito al netto della fiscalità} \\
k_E &=& 0.0650 \quad \mbox{costo del capitale proprio. E’ il tasso di sconto “r”} \\
D &=& 0.5 \mbox{valore del debito gravato da interessi} \\
E &=& 0.5 \mbox{valore del patrimonio netto} 
\end{eqnarray*}
Si è ipotizzato di avere il 50\% delle fonti di finanziamento rappresentati da
capitale proprio, e il restante 50\% da un finanziamento. Quindi $\frac{D}{D+E} 
= \frac{E}{D+E} = 0.5$.  $k_E$ come già calcolato è pari a 4.75\%, mentre per il
costo del debito visto l’elevato rischio di una azienda “neonata” al primo anno
di attività si è imposto $k_D = 6.5\%$. 
Si ottiene un WACC = 0.056
