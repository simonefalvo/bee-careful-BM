\section{Analisi degli investimenti}
\subsection{Tasso di sconto} 
Il tasso di sconto è stato calcolato utilizzando il modello CAPM - Capital Asset
Pricing Model, secondo cui il tasso di sconto è definito come
\begin{displaymath}
r = r_f + \beta (r_m - r_f )
\end{displaymath}
\begin{eqnarray*}
r_f &=& 0.0357 \qquad \mbox{tasso di rendimento di un titolo privo di rischi} \\ 
\beta &=& 1,24 \quad\qquad \mbox{fattore di rischio} \\
r_m &=& 0.0452 \qquad \mbox{tasso di rendimento medio del mercato}
\end{eqnarray*}

Da un’analisi del mercato italiano tecnologico abbiamo trovato come valori
$\beta=1,24$ e $r_m=4.52\%$. Come titolo risk-free è stato scelto il BTP, il cui
tasso di rendimento a 5 anni è $r_f=3.57\%$ (data di riferimento: 15/01/2019).\\
Il valore ottenuto è $r = 4.75\%$\\
Con tale tasso di sconto è possibile determinare dai flussi di cassa gli indici
VAN e TIR.

\url{https://finanza-mercati.ilsole24ore.com/azioni/indici/indici-italia-settoriali/main.php}

\subsection{Stima della domanda}
Si ipotizzata una domanda con crescita secondo una curva logistica del tipo: 
\begin{displaymath}
q(i) = \frac{1}{1 + \beta_1 e^{-i\beta_2}}
\end{displaymath}
Il numero di utenti potenziali è u=50,236 (fonte).
Avendo ipotizzato una crescita con curva logistica, poniamo $q(1)=1\%$ e
$q5=5\%$.

ARPU = ricavi medi per unità =

In un periodo di riferimento i flussi di cassa sono determinati dai ricavi
dovuti all’acquisto del prodotto da parte di nuovi clienti, pertanto gli utenti
che contribuiscono a tale calcolo vengono calcolati come $q(i) - q(i-1)$ per il
periodo $i$ di riferimento.

[TABELLA]

\subsection{VAN e TIR}
I flussi di cassa (guadagno netto - costi ricorrenti) dall’anno t=0 a t=5 sono
rappresentati nella tabella insieme al VAN cumulativo.
Il VAN è calcolato come 	
\begin{displaymath}
VAN = - C_0 + \sum_{i=1}^n \frac{F_i}{(1 + r)^i}
\end{displaymath}
\begin{eqnarray*}
C_0 &=& 80000 \ \mbox{\euro} \qquad \mbox{investimento iniziale} \\
r &=& 0.0475 \qquad \mbox{tasso di sconto} \\
n &:=& \mbox{numero di periodi analizzati} \\
F_i &:=& \mbox{flusso di cassa relativo al periodo i}
\end{eqnarray*}

\subsection{Valutazione del rischio}
I rischi individuati per il prodotto in relazione alla loro natura sono:
\begin{itemize}
\item [rischi puri:] incendi, terremoti, guasti nelle attrezzature, ...
\item [rischi speculativi:] oscillazione dei tassi di cambio, innovazioni
tecnologiche (utilizzo di droni), ...
\end{itemize}

Per osservare l’incidenza di ogni fattore di rischio si fissata una
percentuale di variazione dei parametri associati pari al 10\%. Si è quindi
calcolato il VAN sia nel caso in cui tale variazione sia positiva che negativa,
e si è determinata la sua variazione percentuale rispetto al VAN ottenuto nel
caso “reale”.

I risultati sono rappresentati nel “diagramma tornado” in figura X e nella
tabella Y, che mostrano i fattori di rischio ordinati in base alla loro
incidenza sul VAN. I fattori di rischio con incidenza maggiore sono...

[TABELLA E DIAGRAMMA TORNADO]

%%
%%
\subsection{Modello WACC}

\begin{displaymath}
WACC = k_D \frac{D}{D+E} + k_E \frac{E}{D+E}
\end{displaymath}
\begin{eqnarray*}
k_D &=& 0.0475 \quad \mbox{costo del debito al netto della fiscalità} \\
k_E &=& 0.0650 \quad \mbox{costo del capitale proprio. E’ il tasso di sconto “r”} \\
D &=& 0.5 \mbox{valore del debito gravato da interessi} \\
E &=& 0.5 \mbox{valore del patrimonio netto} 
\end{eqnarray*}

Si è ipotizzato di avere il 50\% delle fonti di finanziamento rappresentati da
capitale proprio, e il restante 50\% da un finanziamento. Quindi $\frac{D}{D+E} 
= \frac{E}{D+E} = 0.5$.  $k_E$ come già calcolato è pari a 4.75\%, mentre per il
costo del debito visto l’elevato rischio di una azienda “neonata” al primo anno
di attività si è imposto $k_D = 6.5\%$. 
Otteniamo un WACC = 0.056
