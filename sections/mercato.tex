\section{Analisi del mercato}
Uno degli obiettivi principali dell'azienda è quello di creare un prodotto che
possa differenziarsi da quelli già presenti sul mercato dell'Hive-Tech.
Questi sono orientati ad offrire sistemi di monitoraggio più o meno complessi
che, grazie alla presenza di vari sensori, possono dare informazioni in tempo
reale sulla popolosità della colonia e la sua produzione, sulla temperatura e
sulle condizioni meteo. L’idea è quindi quella di puntare ad una fetta del
mercato ancora poco esplorata in modo da avere possibilità maggiori di successo
concentrandoci sulla cura delle api attaccate dal parassita ``Varroa
Destructor''.
Bee Careful vuole offrire una soluzione ai gravi problemi che questo parassita
ha creato negli ultimi anni agli apicoltori ed alle aziende di tutto il mondo,
iniziando ad agire dapprima sul territorio nazionale.

Per avere un’idea della struttura del mercato (lato offerta), in assenza di dati
relativi al valore dello stesso, essendo il settore molto giovane, si è tentato
di effettuare un’approssimazione sulle quote di mercato dei potenziali
competitor confrontando i relativi fatturati dell’ultimo anno.

Le principali aziende concorrenti sono 3Bee s.r.l. e Melixa s.r.l., che dai dati
ottenuti dalla banca dati AIDA\cite{aida}, risultano aver fatturato
rispettivamente \EUR{40000}  e \EUR{38000} nel 2018, pertanto una stima per le
quote di mercato è la seguente:
\begin{displaymath}
	s_{3Bee} = 0.5128   \qquad \quad s_{Melixa} = 0.4872
\end{displaymath}
con un indice di Hirschmann-Herfindahl pari a:
\begin{displaymath}
	HHI = s^2_{3Bee} + s^2_{Melixa} = 0.2629 + 0.2373 = 0.5002
\end{displaymath}
Si può quindi ritenere che la struttura del mercato sia associabile ad una
concorrenza perfetta con forte concentrazione e bassa competizione, essendo $HHI
> 0.25$.
