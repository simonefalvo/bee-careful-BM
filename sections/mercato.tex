\section{Analisi del mercato}
Quello dell'``Hive-Tech'' è un mercato in forte crescita sia a livello nazionale
che a livello mondiale, il settore riguarda l’applicazione di soluzioni
tecnologiche come ausilio all’apicoltura.

Come inizio si è deciso di operare in ambito nazionale per poi eventualmente
effettuare una graduale espansione. Questo mercato in Italia conta un bacino di
utenza di circa 50236 apicoltori (dato stimato dall’Osservatorio Nazionale del
Miele~\cite{miele}), e si conta che siano tutti potenzialmente interessati al
prodotto poiché offre una soluzione ad un problema irrisolto che comporta forti
perdite economiche e non richiede una spesa estremamente alta.

Uno degli obiettivi principali dell’azienda è quindi quello di creare un
prodotto che possa differenziarsi da quelli già presenti. Questi sono orientati
ad offrire sistemi di monitoraggio più o meno complessi che, grazie alla
presenza di vari sensori, possono dare informazioni in tempo reale sulla
popolosità della colonia e la sua produzione, sulla temperatura e sulle
condizioni meteo. L’idea è quella di puntare ad una segmento del mercato ancora
poco esplorato in modo da avere possibilità maggiori di successo, concentrandosi
sulla cura delle api dal parassita “Verroa Destructor”, piuttosto che sul
monitoraggio.

Le aziende che operano nel settore sono molto piccole (spesso si tratta di
startup), non è quindi semplice stimare il valore del mercato e le relative
quote. Si può tuttavia affermare in maniera qualitativa che il mercato è
altamente frammentato, essendoci un numero di piccole  aziende concorrenti
nell’ordine delle decine. La maggior parte di esse sono straniere ed operano a
livello globale, non essendoci vincoli o normative che limitino l’attività di
vendita. Inoltre, vi è un’ampia diversificazione dei prodotti in base alle
funzioni svolte ed ai sensori che vi sono installati: si va dalla BuzzBox Mini
di OSBeehives al costo di poche centinaia di dollari al Melixa System che va
oltre la soglia dei 1000 \euro.

Nella tabella~\ref{tab:concorrenza} è mostrato un elenco di alcune concorrenti
nel mercato dell’hive-tech con i prezzi del prodotto che offrono.
\begin{table}[!h]
\centering
%\begin{adjustbox}{width=\textwidth}
\begin{tabular}{c|r@{.}l}
Azienda
& \multicolumn{2}{|c}{Prezzo [\euro]}
\\          
\hline
3bee          & 365    & 00  \\
Melixa        & 1523   & 00  \\
ARNIA         & 331    & 00  \\
OSBeehives    & 173    & 55  \\
Solution bee  & 253    & 67  \\
Label Abeille & 768    & 00  \\
Nectar        & 768    & 00  \\
Pollenity     & 611    & 00 
\end{tabular}
%\end{adjustbox}
\caption{Prezzi concorrenza}
\label{tab:concorrenza}
\end{table}

Altre aziende~\cite{concorrenza} sono ad esempio \emph{Apis}, \emph{IoBee}, 
\emph{The Bee Coorp}, \emph{BuzzTech}, \emph{ApisProtect}, \emph{Beezinga}.      
