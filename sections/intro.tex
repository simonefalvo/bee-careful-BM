\section{Introduzione}
Bee Careful s.r.l.\footnote{Bee Careful s.r.l.: azienda fittizia, nome di
fantasia} è un’azienda che opera nel cosiddetto settore dell’hive-tech, ovvero
tutto ciò che riguarda i servizi e i prodotti tecnologici a supporto
dell’apicoltura. In particolare si propone di realizzare un prodotto che
consenta di limitare e possibilmente debellare la minaccia del parassita “Varroa
Destructor”, per cui tuttora non esiste una soluzione efficace e definitiva.

Il prodotto Bee Careful è un box di espansione applicabile all’ingresso
dell’arnia che rileva, tramite un sistema di riconoscimento di immagini, la
presenza del parassita sulle singole api e conseguentemente, in caso di
contagio, le instrada verso un condotto in cui viene applicato il
trattamento di cura. Quest’ultimo consiste nella somministrazione di acido
ossalico sublimato in maniera capillare sulle singole api, in modo da garantire
efficacia e al tempo stesso impedire l’indebolimento complessivo dello sciame
che scaturirebbe da una  somministrazione di massa.

La situazione attuale del mercato indica una forte concentrazione nelle mani di
due venditori (3Bee e Melixa), tuttavia è anche vero che sono le uniche aziende
in Italia ad operare nel settore, perciò, con il prodotto Bee Careful e le sue
caratteristiche aggiunte innovative, non è escluso che si possa competere alla
pari con esse.

I costi del progetto prevedono un significativo investimento iniziale, ma
apparte questo, i costi ricorrenti riguardanti la realizzazione del prodotto e
tutti gli altri fattori a contorno sono tutto sommato limitati. Si conta di
trarre buoni profitti in confidenza del fatto che si offre una soluzione
economica ad un problema di grandissimo impatto per i diretti interessati, e per
cui la concorrenza non ha ancora fornito la propria soluzione in termini di
cura, ma si limita (per il momento) soltanto al monitoraggio.

Nel seguito del documento viene mostrata l’analisi di tutte le informazioni
necessarie a valutare la potenziale redditività del progetto, valutando la
struttura del mercato, costi, profitti e potenziale remunerativo.

