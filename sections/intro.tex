\section{Introduzione}
Bee Careful s.r.l.\footnote{Bee Careful s.r.l.: azienda fittizia, nome di
fantasia} è un’azienda che opera nel cosiddetto settore dell’hive-tech, ovvero
tutto ciò che riguarda i servizi e i prodotti tecnologici a supporto
dell’apicoltura. In particolare si propone di realizzare un prodotto che
consenta di limitare e possibilmente debellare la minaccia del parassita “Varroa
Destructor”, per cui tuttora non esiste una soluzione efficace e definitiva.

Il prodotto Bee Careful è un box di espansione applicabile all’ingresso
dell’arnia che rileva, tramite un sistema di riconoscimento di immagini, la
presenza del parassita sulle singole api e conseguentemente, in caso di
contagio, le instrada verso un condotto in cui viene applicato il
trattamento di cura. Quest’ultimo consiste nella somministrazione di acido
ossalico sublimato in maniera capillare sulle singole api, in modo da garantire
efficacia e al tempo stesso impedire l’indebolimento complessivo dello sciame
che scaturirebbe da una somministrazione di massa.

Il presente documento mostra lo studio di fattibilità e la potenziale
remunerabilità del progetto analizzando la struttura del mercato, costi e
profitti legati ad esso.

Il mercato di riferimento è molto eterogeneo, composto da aziende che offrono
diverse tipologie di prodotto che per la maggior parte si concentrano sul
monitoraggio delle arnie, lo scopo è quindi di puntare a quel segmento di
mercato “inesplorato” riguardante il trattamento contro il parassita sopra
citato.

Si conta di trarre buoni profitti entro pochi anni di esercizio in confidenza
del fatto che si offre, a costi contenuti, una soluzione ad un problema di
grandissimo impatto economico per i diretti interessati, per il quale la
concorrenza non ha ancora fornito una propria soluzione.
