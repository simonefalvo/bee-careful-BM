\section{Analisi dei costi}
L'analisi dei costi è stata effettuata impiegando un metodo di stima analitico,
poiché, essendo una proposta innovativa, non è stato possibile reperire dati
storici riguardanti i costi di progetti simili.
%
%
\subsection{Composizione del prodotto}
\subsubsection{Componenti hardware}
\begin{itemize}
\item \textbf{1 Arduino}: scheda elettronica dotata di microcontrollore per
attuare la logica di controllo dell’intero sistema
\item \textbf{4 Microcamere}: consentono la scansione delle immagini per il
riconoscimento delle api che ospitano il parassita
\item \textbf{2 Fotoresistori}: permettono di tenere il conto delle api passate
sotto la microcamera, poiché questa non vede quelle senza varroa.
\item \textbf{1 Servomotore}: permette di indirizzare le api nell’arnia o nel
condotto di trattamento.
\item \textbf{1 Resistenza a cartuccia}: consente la sublimazione dell’acido
ossalico, è a base di nichel cromo, resistente all’ossidazione ed alla
corrosione.
\item \textbf{1 Sensore di temperatura}: consente il controllo della temperatura
del resistore
\item \textbf{1 Amplificatore di corrente}: fornisce la corrente di servizio
necessaria ai componenti elettrici
\item \textbf{1 Colorante gulal}: le api trattate vengono marcate in modo da non
venire ritrattate a breve termine
\item \textbf{1 Box contenitore}: struttura che contiene il prodotto
\end{itemize}
%
\subsubsection{Componenti software}
\begin{itemize}
\item \textbf{Programma di controllo}: Programma che pilota i dispositivi
hardware e descrive la logica di acquisizione dati, elaborazione ed attuazione
\end{itemize}
%
%
\subsection{Fasi di sviluppo del progetto}
\begin{enumerate}
\item Realizzazione del software\footnote{Operazione Una Tantum}
\item Inserimento in sede\footnote{Operazione Una Tantum}
\item Acquisto materie prime
\item Assemblaggio materie prime e caricamento del software
\item Vendita e spedizione del prodotto
\item Assistenza
\end{enumerate}
%
%
\subsection{Stima delle risorse e dei costi}
Le risorse necessarie alla realizzazione del progetto vengono classificate e
quantificate  seguendo una ripartizione tale da individuare costi fissi e
variabili.  
%
\subsubsection{Costi fissi}
I costi fissi sono tutti quei costi che non variano a seconda della domanda.

I costi sono stati classificati in “diretti” e “indiretti”. I costi diretti
riguardano le risorse umane o materiali necessarie esclusivamente alla
realizzazione del prodotto (materie prime, manodopera diretta, acquisto di beni
e servizi da terzi). I costi indiretti sono legati a manutenzione, ammortamenti,
energia e costi generali.\\
%
Sono stati individuati i seguenti costi fissi diretti:
\begin{itemize}
\item \textbf{Programmatori (\EUR{3200})}: risorse umane necessarie alla
realizzazione del software.  Le funzionalità del software riguardano
principalmente acquisizione dei dati, l’elaborazione, ed il controllo dei
dispositivi connessi alla scheda Arduino.  Per quanto riguarda l’acquisizione
dei dati tramite riconoscimento immagini, è già disponibile una libreria
software open source (OpenCV~\cite{opencv}) integrabile con la scheda, pertanto
si può considerare soltanto il costo di adattamento della libreria. Anche per
quanto riguarda le funzioni di controllo e attuazione, vi è un elevato grado di
supporto in casa Arduino, pertanto si stima un tempo necessario per
l’implementazione ed il test dell’ordine delle settimane, non più di 4, pari a
160 ore di lavoro. Si può concludere che il costo totale sia al più di
\EUR{3200} per un impiego su commissione, supponendo che un programmatore lavori
per \EUR{20} l’ora (lordi).  
\item \textbf{Computer (\EUR{1400})}: risorse materiali
necessarie per la gestione degli aspetti commerciali, contabilità, fatturazione
ed assistenza, si prevede l’acquisto di 2 unità per un costo di \EUR{700} a
macchina.  
\item \textbf{Dipendenti (\EUR{4000}/mese)}: risorse umane con competenze
tecniche atte allo svolgimento delle mansioni necessarie all’attività
dell’azienda, si prevede un impiego fisso per 8 ore al giorno per un totale di
\EUR{2000} lordi mensili per dipendente.  
\end{itemize}
%
I costi fissi indiretti sono: 
\begin{itemize}
\item \textbf{Affitto immobile (\EUR{1000}/mese)}: locale comprendente un
magazzino per la giacenza dei prodotti e delle materie materie prime,
laboratorio di produzione, ufficio e bagno;
\item \textbf{Energia elettrica (\EUR{70}/mese)}
\item \textbf{Acqua (\EUR{30}/mese)}
\item \textbf{Gas (\EUR{100}/mese)}
\item \textbf{Telefono (\EUR{30}/mese)}
\end{itemize}
%
Le utenze di gas ed energia elettrica sono aggiornate al mese di dicembre 2018,
come riportato da Federconsumatori in~\cite{feder}.

L’ammontare totale dei costi fissi $\Phi$ è dato dalla somma delle spese mensili
ricorrenti:
\begin{displaymath}
\Phi = (4000 + 1000 + 70 + 30 + 100 + 30) \ \mbox{\euro/mese} = 3230 \ 
\mbox{\euro/mese}
\end{displaymath}
Nel calcolo non sono stati inseriti il costo dei computer e del software in
quanto non sono spese mensili ma effettuate solo all’inizio dell’attività.
Saranno ammortizzate nel bilancio annuale secondo le norme vigenti in Italia.
%
\subsubsection{Costi variabili}
I costi variabili sono invece tutti quei costi che variano a seconda della
domanda e quindi del volume di produzione.

La società dovrà acquistare i sensori per il monitoraggio dell’arnia e le
componenti per la somministrazione della cura alle api. Inoltre si aggiungono al
conto le spese di spedizione, in quanto il prodotto viene inviato al cliente che
provvede da solo all’installazione.\\
%
Di seguito i costi per componente elementare:
\begin{itemize}
\item \textbf{Arduino: \EUR{35}}
\item \textbf{Microcamere: \EUR{60}}
\item \textbf{Fotoresistori: \EUR{1}}
\item \textbf{Servomotore: \EUR{2}}
\item \textbf{Resistenza: \EUR{12}}
\item \textbf{Sensore di temperatura: \EUR{2}}
\item \textbf{Amplificatore di corrente: \EUR{3}}
\item \textbf{Colorante: \EUR{13}}
\item \textbf{Box: \EUR{25}}
\item \textbf{Spese di trasporto: \EUR{20}} (prezzo medio per spedire in tutta
Italia un pacco del peso di circa 5 Kg)
\end{itemize}
L’ammontare dei costi variabili $\mu$ è
\begin{displaymath}
\mu= (35 + 60 + 1 + 2 + 12 + 2 + 3 + 13 + 25 + 20) \ \mbox{\euro} = 173
\ \mbox{\euro}
\end{displaymath}
Il costo finale per la produzione di un singolo prodotto è di $\mu$ = \EUR{173}.
%%
%%
\subsection{Ripartizione dei costi}
Vengono interpretati come CAPEX tutti i costi relativi a capitali che rimangono
della proprietà dell’azienda, quindi i costi di sviluppo software e l’affitto
dei terminali, mentre vengono interpretati come OPEX tutti i costi relativi
all’affitto degli uffici e alle singole installazioni.




%%
%%
\subsection{Fattore di markup e prezzo ottimo}
Non essendo ancora in possesso di dati sperimentali sulla risposta della domanda
in funzione del prezzo, si è ipotizzata una relazione funzione di potenza del
tipo: 
\begin{displaymath}
q = \alpha p^{-\beta}
\end{displaymath}
dove $\beta$ corrisponde all’elasticità della domanda.\\
Considerando una reazione significativa del consumatore al variare del prezzo
dovuta al fatto che si è un’azienda appena nata, il parametro $\beta$ è stato
fissato ragionevolmente ad un valore pari a 2, per cui si ottiene un fattore di
markup $MRP=2$, pertanto il valore del prezzo ottimo finale è dato dalla
seguente formula:
\begin{displaymath}
	p^* = MRP \cdot \mu =  2 \cdot 173 \ \mbox{\euro} \ = 346 \ \mbox{\euro}
\qquad\quad    MRP = \frac{\beta}{\beta-1} = 2
\end{displaymath}
%%
%%
\subsection{Break even point}
L’analisi del “punto di pareggio” è stata eseguita per determinare il volume di
produzione necessario per coprire i costi del prodotto.  Il punto di break-even
$q^*$ è dato dall’intersezione tra la retta dei costi e quella dei ricavi. 

\setlength\arraycolsep{2pt}
\begin{displaymath}
\left\{ \begin{array}{rl}
c &= \phi + \mu q \\
r &= q p^* 
\end{array} \right .
\end{displaymath}
%
\begin{eqnarray*}
\phi &=& 3230 \cdot 12 = 38.760 \ \mbox{\euro/anno} \qquad \mbox{costi fissi
annuali} \\
p^* &=& 346 \ \mbox{\euro/prodotto} \ \qquad\quad\qquad\quad \mbox{prezzo ottimo} \\
\mu &=& 173 \ \mbox{\euro/prodotto} \ \qquad\quad\qquad\quad \mbox{costo variabile unitario}
\end{eqnarray*}
%
\begin{displaymath}
q^* = \frac{\phi}{p^* - \mu} = \frac{38.760}{346 - 173} = 224.05
\end{displaymath}
Quindi la vendita del duecentoventiquattresimo prodotto ci permetterà di
arrivare al “punto di pareggio” tra costi e ricavi, mostrato in figura X. Oltre
tale soglia di vendita si avrà un guadagno positivo.

GRAFICO

