\section{Conclusioni}
Dalla descrizione del prodotto e del relativo processo produttivo siamo arrivati
ad una stima dei costi (fissi e variabili), ottenendo che il costo per un
singolo prodotto è di \EUR{145.3}. Da tale costo è stato deciso il prezzo di
vendita pari a \EUR{350.00}, derivante da un trade-off tra proposizione di
valore del prodotto prezzo imposto dalla concorrenza sul mercato.

L’analisi del punto di pareggio ci ha permesso di dedurre che la vendita della
duecentonovantottesima unità del prodotto ci porterà a pareggiare costi e
ricavi, quindi dalle vendite successive si avranno dei guadagni.

Nell’analisi della domanda abbiamo ipotizzato l’andamento della domanda del
periodo di riferimento dall’anno uno all’anno cinque. Sono stati calcolati
quindi i flussi di cassa, relativi al tasso di sconto r = 9.07\%, da cui è stato
ottenuto un VAN di  \EUR{65687.06} all’anno cinque. Dall’analisi della stima
ottenuta, siamo soddisfatti di mostrare il raggiungimento del pareggio
finanziario a cavallo tra il quarto ed il quinto anno di esercizio.  Il tasso di
rendimento del progetto è stato definito con l’ausilio del TIR, il quale risulta
essere pari al 28.55\% .

Infine per valutare l’incidenza dei singoli fattori sul VAN è stata condotta
un’analisi dei fattori di rischio, la quale ha mostrato che il fattore con incidenza
maggiore è la variazione della domanda.  Poiché il TIR ottenuto è
sufficientemente maggiore del WACC, possiamo concludere che investire nel nostro
prodotto è potenzialmente conveniente.
