\section{Conclusioni}
Dall’analisi del mercato di riferimento si è ottenuto che ha una struttura
associabile ad una concorrenza perfetta con forte concentrazione e bassa
competizione.

Dalla descrizione del prodotto e del relativo processo produttivo si è arrivati
ad una stima dei costi (fissi e variabili), ottenendo che il costo per un
singolo prodotto è di \EUR{173.00}, da cui è stato ricavato il prezzo di
vendita pari a \EUR{323.00} che, maggiorato di un fattore 2 (Markup), ci
permette di ottenere dei ricavi dalla vendita.

Dall'analisi del punto di pareggio è emerso che la vendita della
duecentoventiquattresima unità del prodotto porterà a pareggiare costi e ricavi,
quindi dalle vendite successive si avranno dei guadagni effettivi.

Nell’analisi della domanda si è ipotizzato l’andamento della domanda del
periodo di riferimento dall’anno uno all’anno cinque. Le ipotesi sono basate sul
numero di apicoltori registrati in Italia, che nel Dicembre 2017 è stato stimato
come 50.236. Dopodiché Sono stati calcolati i flussi di cassa, relativi al tasso di sconto r=4.75\%, da
cui è stato ottenuto un VAN = \EUR{13.663,02} all’anno 5, Pertanto si punta a
raggiungere il pareggio finanziario a cavallo tra il quarto ed il quinto anno di
esercizio.

Il tasso di rendimento del progetto è stato definito con l’ausilio del TIR, il
quale risulta essere pari all'8.505\% .

Per valutare l’incidenza dei singoli fattori sul VAN è stata condotta un’analisi
dei rischi associati, la quale ha mostrato che quello con incidenza maggiore è
è legato all'aumento della concorrenza essendo il fattore di Markup il parametro
che fa variare maggiormente il VAN.

Infine sono state valutate le fonti di finanziamento del progetto, divise
tra capitale proprio e finanziamento da terzi, attraverso il modello WACC,
ottenendo un WACC pari al 5.6\%. Poiché il TIR ottenuto è maggiore del WACC,
si può concludere che investire nel prodotto Bee Careful è potenzialmente
conveniente.
